\begin{titlepage}
\begin{center}

%forma de introducir imágenes. el \\[0.5 cm] de final de línea introduce un salto de ese tamaño.
%width=1\textwidth indica el tamaño de la imágen (valores entre 0-1). 
 \includegraphics[width=1\textwidth]{figuras/cabecera.png}  \\[0.5 cm]

\LARGE UNIVERSIDAD POLITÉCNICA DE MADRID \\ [1 cm]

\LARGE ESCUELA TÉCNICA SUPERIOR DE INGENIERÍA Y DISEÑO INDUSTRIAL \\ [1 cm]

\LARGE Grado en Ingeniería Electrónica y Automática Industrial\\ [1 cm]

\LARGE \textbf{TRABAJO FIN DE GRADO}\\[1 cm]

\Huge \textsc{Simulación de juego de póker para la comparación de estrategias de juego}\\[1 cm]

\LARGE Autor: Álvaro González Maestre \\[2 cm]

%flushleft alinea a la izquierda el texto

\begin{multicols}{2} 
\begin{flushleft} \Large
\emph{Cotutor (si lo hay):}  \\
\emph{Departamento:} 
\end{flushleft}

\begin{flushleft} \Large
\emph{Tutor:}Daniel Jeremy Fox Hornig\\
\emph{Departamento:} Matemáticas del Área Industrial
\end{flushleft}

\end{multicols} 

%rellena de blanco el resto de la página para escribir abajo del todo
\vfill

% Bottom of the page
{\large Madrid, Junio, 2020}

%SE ponen al final firmas.tex
%\end{center}
%\end{titlepage}


\cleardoublepage 