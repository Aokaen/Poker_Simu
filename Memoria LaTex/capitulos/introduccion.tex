\chapter{Introducción al proyecto}


\section{Motivaciones detrás del proyecto: gestión de información conocida y juegos de cartas}

La importancia de la gestión de información conocida y estimación de la información desconocida es algo que conozco en primera persona. 

En un mundo teórico e ideal toda la información es conocida, por lo que no hay posibilidad de que ocurran errores. Tampoco habría posibilidad de encontrar un comportamiento anómalo, un resultado inésperado o un mal funcionamiento de cualquier cosa. En este entorno ideal de información conocida, la toma de decisiones es bastante sencilla y simple, incluso puede que se dieran casos en los que sería innecesario tomar una decisión.

Pero, en la práctica, esto no funciona así. Por eso, es necesario hacer frente a esa incertidumbre en, prácticamente, todos los aspectos de la vida, tanto profesional como cotidiana.

La idea inicial que me sirvió de motivación para hacer este proyecto es hacer ver cómo de importante puede llegar a ser la gestión de la información (tanto de la información conocida como la información que es desconcida pero se tiene constancia de que esa información falta) ya que en mi vida laboral he tenido que afrontar en varias ocasiones. 
Tanto durante la etapa que estuve trabajando como técnico de electrónica de soporte como en la que estuve trabajando como ingeniero de pruebas de software, he podido conocer lo importante que es establecer la información conocida, delimitar los factores de riesgo e incertidumbre y tomar decisiones en torno a ambos tipos de informaciones. También pude aprender que, incluso delimitando los posibles riesgos, hay veces que aparecen riesgos que uno ni siquiera había contemplado antes. 

Además de la experiencia en el ámbito laboral, otra motivación que he tenido para realizar este proyecto es mi afición a los juegos de cartas. Los juegos de cartas me han acompañado casi toda la vida, como una afición que he compartido tanto con muchos familiares y con amigos. 

Desde las partidas de Chinchón que jugaba con mis familiares para apostarnos las tareas del hogar durante los veranos que pasábamos juntos hasta los torneos competitivos que he participado (tanto de jugador como de juez) de juegos de cartas coleccionables, pasando por las tardes jugando pachangas con amigos, han hecho de los juegos de cartas una de las aficiones con las que mas he disfrutado (y, en algunas ocasiones, frustrado) durante toda mi vida.
Además de lo mucho que me entretienen los juegos de cartas, gracias a los juegos de cartas coleccionables y a los eventos relacionados con ellos he podido conocer a una gran variedad de personas, inluidas personas muy importantes para mí que han sido (y siguen siendo) uno de los mayores apoyos que tengo a nivel personal.

Estas dos motivaciones han implicado que me decantara por realizar este proyecto.

\section{Objetivos}

El objetivo final de este Trabajo de Fin de Grado es el desarrollo de un algoritmo para la toma de decisiones durante una partida de Póker Texas Hold’em. Para poder llegar a ese objetivo, primero es necesario desarrollar un motor de juego funcional que sea capaz de realizar iteraciones para probar dicho algoritmo, así como realizar la conexión adecuada entre el algoritmo y el motor de juego. Por tanto, los objetivos de este proyecto serían los siguientes:
\begin{itemize}
\item Desarrollo de un motor de juego de Texas Hold’em, con la posibilidad de jugar Persona contra Algoritmo, así como hacer n iteraciones de juego entre el algoritmo diseñado y un patrón de comportamiento predefinido.
\item Desarrollo del algoritmo para la toma de decisiones durante la partida.
\item Creación del enlace entre ambos elementos (motor de juego y algoritmo).
\end{itemize} 

\section{Materiales utilizados}

Este proyecto consiste en elementos de programación, por lo que no se han utilizado materiales fuera de herramientas de compilación y lectores de documentos digitales.

\section{Estructura del documento}

A continuación y para facilitar la lectura del documento, se detalla la estructura de este, así como el contenido de cada uno de los capítulos del proyecto.

\begin{itemize}
\item En el capítulo 1 se realiza una introducción sobre el proyecto, los objetivos del proyecto y la estructura del documento.
\item En el capítulo 2 se repasa la situación actual de la simulación de póker, así cómo de los desarrollos de inteligencia artificial sobre póker.
\item En el capítulo 3 se explican algunas de las diferentes variantes del póker así como las nociones básicas del póker Texas Hold'em
\item En el capítulo 4 se elabora una explicación de las bases matemáticas detrás del póker.
\item En el capítulo 5 se tratarán las bases de programación necesarias, al igual que se hablará de los patrones de comportamiento.
\item En el capítulo 6 se explicará el desarrollo del motor de juego, incluyendo las abstracciones decididas en el proyecto
\item En el capítulo 7 se explicará el desarrollo del algoritmo de toma de decisiones
\item En el capítulo 8 se explicará el desarrollo del enlace entre el motor de juego y el algoritmo
\item En el capítulo 9 se confeccionarán los resultados del algoritmo frente a los patrones de comportamiento con los datos obtenidos.
\item En el capítulo 10 se tratarán los aspectos mas logísticos del proyecto, como la planificación de este.
\item En el capítulo 11 se expondrán las conclusiones a las que se han llegado a raíz de los resultados, así como se plantean mejoras y posibles desarrollos futuros.
\end{itemize}
